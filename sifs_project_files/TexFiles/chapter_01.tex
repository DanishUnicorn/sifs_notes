\chapter*{Course Description}
\setlength{\headheight}{12.71342pt}
\addtolength{\topmargin}{-0.71342pt}

\section*{Content}
The course aims to provide the students with an understanding of sustainable innovation in the food chain and sector; e.g. gentle food processing technologies and circular business models that result in reduced environmental footprints. The students will be introduced to key aspects such as:
\begin{itemize}
    \item Life Cycle Assessment (LCA): In-depth exploration of the four steps of LCA— goal and scope definition, inventory analysis, impact assessment, and interpretation.
    \item Sustainable Innovation and Systems Design: Exploration of circular business models and sustainable production systems designed for resource efficiency and environmental impact reduction.
\end{itemize} 

The students will work theoretically with identifying potential solutions to current challenges in the food chain and sector. The student will work in a group on a relevant challenge that is linked to the teaching on a weekly basis. 

Students will apply theoretical knowledge to practical challenges divided in portfolio parts. They will reflect on real-world restrictions and opportunities, including IPR, food regulation, and data security considerations. 

\section*{Learning Outcome}
\textbf{Knowledge:}
\begin{itemize}
    \item Describe the four steps of Life Cycle Assessment (LCA), including system boundaries and functional units
    \item Explain key concepts of sustainable innovation and circular economy in food systems design.
    \item Identify environmental impacts and potential improvements in food processing and production systems.
    \item Basic insight into IPR, regulation, ethics, and consumer preferences
\end{itemize}

\textbf{Skills:}
\begin{itemize}
    \item Ability to conduct and interpret comparative LCA, including mass balance calculations for the chosen challenges.
    \item Ability to critically reflect on bottlenecks in the implementation of sustainable innovation in the food chain
    \item Ability to apply theoretical LCA knowledge to practical challenges.
    \item Ability to identify business expansion opportunities for sustainable innovation in the food chain.
\end{itemize}

\textbf{Competences:}
\begin{itemize}
    \item Ability to collaborate effectively in interdisciplinary groups to address complex sustainability challenges.
    \item Ability to analyse environmental performance and propose improvements based on LCA results.
    \item Ability to communicate findings and solutions effectively using LCA data and theoretical frameworks.
    \item Ability to contextualize system and product environmental footprint within the broader frameworks addressing the grad societal challenges of UN's SDGs and planetary boundaries in the course challenge.
\end{itemize}


\section*{Litterature}
See Absalon for a list of course literature

\section*{Recommended Academic Qualifications}
Academic qualifications equivalent to a BSc degree is recommended.

We recommend the students have a Windows based operating system for the LCA calculations.


\section*{Teaching and Learning Methods}
Lectures, exercises and group work. The student will work in a group on a relevant challenge documented by a portfolio addressing the key aspects of the course.


\section*{Remarks}
The course is identical to the discontinued course NFOK20004U Short Thematic Course in Food Science and Technology. Therefore you cannot register for NFOK20003U - Sustainable Innovation in Food Science, if you have already passed NFOK20004U Short Thematic Course in Food Science and Technology.

If you are registered with examination attempts in NFOK20004U Short Thematic Course in Food Science and Technology without having passed the course, you have to use your last examination attempts to pass the exam in NFOK20003U - Sustainable Innovation in Food Science. You have a total of three examination attempts.


\section*{Workload}
See table \ref{tab:workload_01} for an overview of the workload for the course. 
\begin{table}[h]
    \centering
    \caption{A table with an overview over the workload for the course.}
    \label{tab:workload_01}
    \rowcolors{2}{white}{gray!7}
    \begin{tabular}{ l | c}
        \textbf{Category} & \textbf{Hours} \\ 
        \hline
        Lectures & 54 \\ 

        Preparation & 42 \\

        Theory exercises & 30 \\ 

        Project work & 73 \\

        Guidance & 6 \\

        Exam & 1 \\ 
        \hline
        Total & 206 \\ 
    \end{tabular}
\end{table}

\section*{Feedback format}
Written
Oral
Collective
Continuous feedback during the course of the semester

\section*{Exam}
See table \ref{tab:exam_info} for an overview of the exam.
\begin{table}[h]
    \centering
    \caption{A table with an overview over the the examination for the course.}
    \label{tab:exam_info}
    \rowcolors{2}{white}{gray!7}
    \begin{tabular}{ p{0.15\textwidth} | p{0.85\textwidth} }
        \textbf{Category} & \textbf{Hours} \\ 
        \hline
        Credit & 7,5 ECTS \\ 

        Type of assessment & Oral examination, 15 minutes \\

        Type of assessment details & The exam is an individual oral exam (15 minutes) without preparation time based on both previous submissions and the course curriculum.

        Written assignments are done in groups and consist of multiple
        parts with deadlines throughout the course.
        
        In order to access the oral exam the submissions (written assignments) must be submitted during the course. \\ 

        Examination prerequisites & Use of AI is only allowed for the written assignments as described in the challenge guidelines provided on the course.  \\

        Aid & Only certain aids allowed (see description below)
        The student can bring a paper version of the challenge portfolio parts to the oral exam. \\

        Marking scale & 
        7-point grading scale \\ 

        Censorship form & External censorship \\
        
        Re-exam & Same as ordinary exam.

        In order to access the oral exam the submissions (written assignments) must be submitted two weeks before the oral examination.
        
        Any previously submitted submissions (written assignments) will be reused. \\
    \end{tabular}
\end{table}



\newpage