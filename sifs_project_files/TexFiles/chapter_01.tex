\chapter*{Course Description}
\setlength{\headheight}{12.71342pt}
\addtolength{\topmargin}{-0.71342pt}

\section*{Content}
The course aims to provide the students with an understanding of sustainable innovation in the food chain and sector; e.g. gentle food processing technologies and circular business models that result in reduced environmental footprints. The students will be introduced to key aspects such as:
\begin{itemize}
    \item Life Cycle Assessment (LCA): In-depth exploration of the four steps of LCA— goal and scope definition, inventory analysis, impact assessment, and interpretation.
    \item Sustainable Innovation and Systems Design: Exploration of circular business models and sustainable production systems designed for resource efficiency and environmental impact reduction.
\end{itemize} 

The students will work theoretically with identifying potential solutions to current challenges in the food chain and sector. The student will work in a group on a relevant challenge that is linked to the teaching on a weekly basis. 

Students will apply theoretical knowledge to practical challenges divided in portfolio parts. They will reflect on real-world restrictions and opportunities, including IPR, food regulation, and data security considerations. 

\section*{Learning Outcome}
\textbf{Knowledge:}
\begin{itemize}
    \item Describe the four steps of Life Cycle Assessment (LCA), including system boundaries and functional units
    \item Explain key concepts of sustainable innovation and circular economy in food systems design.
    \item Identify environmental impacts and potential improvements in food processing and production systems.
    \item Basic insight into IPR, regulation, ethics, and consumer preferences
\end{itemize}

\textbf{Skills:}
\begin{itemize}
    \item Ability to conduct and interpret comparative LCA, including mass balance calculations for the chosen challenges.
    \item Ability to critically reflect on bottlenecks in the implementation of sustainable innovation in the food chain
    \item Ability to apply theoretical LCA knowledge to practical challenges.
    \item Ability to identify business expansion opportunities for sustainable innovation in the food chain.
\end{itemize}

\textbf{Competences:}
\begin{itemize}
    \item Ability to collaborate effectively in interdisciplinary groups to address complex sustainability challenges.
    \item Ability to analyse environmental performance and propose improvements based on LCA results.
    \item Ability to communicate findings and solutions effectively using LCA data and theoretical frameworks.
    \item Ability to contextualize system and product environmental footprint within the broader frameworks addressing the grad societal challenges of UN's SDGs and planetary boundaries in the course challenge.
\end{itemize}
 




\section*{Litterature}
See Absalon for a list of course literature. The curriculum will include lecture notes and scientific papers in addition to parts of textbooks.

\section*{Recommended Academic Qualifications}
Qualifications corresponding to having completed the Dairy Internship are recommended.

Academic qualifications equivalent to a BSc degree is recommended .


\section*{Teaching and Learning Methods}
Lectures, project work in the dairy pilot plant and laboratory and half/one-day excursions.

Minor costs for one-day excursions (e.g. transportation) are paid by the students.

\section*{Workload}

\begin{table}[h]
    \centering
    \caption{A table with an overview over the workload for the course.}
    \label{tab:workload}
    \rowcolors{2}{white}{gray!7}
    \begin{tabular}{ l | c}
        \textbf{Category} & \textbf{Hours} \\ 
        \hline
        Lectures & 30 \\ 

        Preparation & 88 \\

        Practical exercises & 12 \\ 

        Excursions & 8 \\

        Project work & 55 \\

        Guidance & 12 \\

        Exam & 1 \\ 
        \hline
        Total & 206 \\ 
    \end{tabular}
\end{table}





\newpage